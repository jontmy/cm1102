\section{Molecular Structure}
Any object that is heated to high temperatures \textit{emits electromagnetic radiation}.

The wavelength emitted depends on the temperature, by \textbf{Wien's displacement law}:

\begin{tblr}{|l|X|l|} \hline
    \SetCell[c=3]{c} $\lambda_\text{max} = \frac{b}{T}$ \\ \hline
    & \textbf{value} & \textbf{unit} \\ \hline
    $\lambda_\text{max}$ & peak wavelength & \si{\metre} \\ \hline[dashed]
    $b$ & {Wien's displacement constant \\ $\approx 2.898 \times 10^{-3}$} & \si{\metre\kelvin} \\
    $T$ & absolute temperature & \si{\kelvin} \\ \hline
\end{tblr}

\textbf{Ideal blackbody radiators} are objects which are \textit{perfect absorbers and emitters} of radiation.

Absorption of IR radiation causes \textbf{molecular vibrations}
--- \textit{bending} and \textit{stretching}.

Assymetric stretching has a higher vibrational frequency than symmetric stretching due to preservation of momentum.


\subsection{Lewis Structure}
\textbf{Covalent bonds} are formed when two neighboring atoms \textit{share a pair of electrons}.

Only \textbf{valence electrons} --- the electrons in the outermost shell --- are involved;
valence electrons that are \textit{not involved} are known as \textbf{lone-pair electrons}.

Lone-pair electrons determine the shape and chemical properties of a molecule.

Stability of compounds occurs when atoms achieve \textbf{noble gas electronic configurations},
but there are exceptions to the \textbf{octet rule}:
\begin{itemize}
    \item the \textbf{duet rule}: \ce{H2},
    \item the \textbf{incomplete octet}: \ce{BF3}, and,
    \item the \textbf{hypervalent molecules}: \ce{SF6}, \ce{PCl5}.
\end{itemize}

Drawing Lewis structures:
\begin{enumerate}[leftmargin=*]
    \item Find the total number of valence electrons.
    \item Place the least electronegative atom in the center.
    \item Draw a single bond from each surrounding atom to the central atom.
    \item Find the remaining number of electrons.
    \item Arrange the remaining electrons as lone pairs or multiple bonds to satisfy the octet rule.
    \item Assign charges (if any).
\end{enumerate}

\subsubsection{Resonance}
\textbf{Resonance structures} are different Lewis structures for the \textit{same molecule}
--- they have the\textit{ same relative positions of atoms}, 
but \textit{different bonding and lone-pair electrons}.

\textbf{Resonance hybrids} has a \textit{real structure} of its \textit{average resonance structures}.
They have lower enegy that any of its contributing structure.

\subsubsection{Formal Charge}
\textit{The extent to which an atom has gained or lost an electron while forming covalent bonds.}

\begin{tblr}{|l|X|l|} \hline
    \SetCell[c=3]{c} $FC = V - L - \frac{1}{2}B$ \\ \hline
    & \textbf{value} & \textbf{unit} \\ \hline
    FC & formal charge & \code{int} \\ \hline[dashed]
    V & number of valence electrons & \code{int} \\
    L & number of lone-pair electrons & \code{int} \\
    B & number of bonding electrons & \code{int} \\ \hline
\end{tblr}

The \textit{sum of formal charges} equals the \textit{actual charge} of the molecule.

The most important resonance structure:
\begin{itemize}
    \item minimizes \textit{formal charges},
    \item minimizes \textit{separation of charges}, and,
    \item places the negative formal charge on the \textit{more electronegative elements}.
\end{itemize}


\subsection{VSEPR Model}
The \textit{valence-shell electron-pair repulsion} (VSEPR) model determines the structure
around an atom by \textit{minimizing electron-pair repulsions}.

The strength of electron-pair repulsion depends on the pair (in decreasing order of strength):
\begin{itemize}
    \item lone-pair -- lone-pair
    \item lone-pair -- bonding pair
    \item bonding pair -- bonding pair
\end{itemize}

Determining molecular structure:
\begin{enumerate}[leftmargin=*]
    \item Draw the \textit{Lewis structure}.
    \item Arrange electron groups to \textit{minimize repulsion}.
    \item Identify the molecular structure from the \textit{position of atoms}.
\end{enumerate}

\subsubsection{Molecular Polarity}
The \textbf{molecular shape} and \textbf{bond polarity} determine \textit{molecular polarity},
which in turn influences its physical and chemical properties.

\textbf{Electronegativity} is the tendency for an atom to attract shared electrons when forming
a chemical bond.

\begin{itemize}[label=\(\implies\)]
    \item \textbf{Electron density} is pulled towards the more electronegative atoms.
    \item \textbf{Dipole moments} (\(\mu\)) are generated, but can cancel each other, e.g. \ce{CO2}.
    \item The overall dipole moment affects the \textbf{molecular polarity}, cf. \textbf{cis-trans isomers}.
\end{itemize}

Change in dipole moment when the molecule vibrates causes \textbf{infrared activity}.