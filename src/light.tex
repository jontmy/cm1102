\section{Light}
\textbf{Light} is a form of energy that consists of oscillating \emph{electric} and \emph{magnetic}
fields, perpendicular to each other.

\begin{tabularx}{\linewidth}{|l|X|l|} \hline
    \multicolumn{3}{|c|}{$c = \lambda \nu$} \\ \hline
    & \textbf{value} & \textbf{unit} \\ \hline
    $c$ & speed of light $= 2.9979 \times 10^8$ & m/s \\ \hdashline
    $\lambda$ & wavelength & m \\
    $\nu$ & frequency & $s^{-1}$ or Hz \\ \hline
\end{tabularx}

\subsection{Wave-particle Duality of Light}

Due to the wave nature of light, light waves that are \textit{in phase} undergo 
\textbf{constructive interference}.

Light waves that are \textit{out of phase} undergo \textbf{destructive interference} 
--- cancelling out each other.

Due to the particle nature of light, when an object is heated, 
it emits electromagnetic radiation known as \textbf{blackbody radiation}.

The wavelength of light emitted changes as the temperature of the object increases.

\begin{tabularx}{\linewidth}{|l|X|l|} \hline
    \multicolumn{3}{|c|}{$E = h \nu$} \\ \hline
    & \textbf{value} & \textbf{unit} \\ \hline
    $E$ & energy & J \\ \hdashline
    $h$ & Planck's constant $= 6.626 \times 10^{-34}$& J/s \\
    $\nu$ & frequency & $s^{-1}$ or Hz \\ \hline
\end{tabularx}

Putting the two equations above together, we can interconvert between \textit{energy}, 
\textit{frequency}, and \textit{wavelength}.

\begin{tabularx}{\linewidth}{|l|X|l|} \hline
    \multicolumn{3}{|c|}{$E = \frac{hc}{\lambda}$} \\ \hline
    & \textbf{value} & \textbf{unit} \\ \hline
    $E$ & energy & J \\ \hdashline
    $h$ & Planck's constant $= 6.626 \times 10^{-34}$& J/s \\
    $c$ & speed of light $= 2.9979 \times 10^8$ & m/s \\
    $\lambda$ & wavelength & m \\ \hline
\end{tabularx}

\subsection{Photoelectric Effect}
Electrons can only be ejected from a surface if incoming \textbf{photons} transfer
sufficient energy to the atoms on the metal surface.

\textbf{Electron-volt} (eV) is the amount of \textit{kinetic energy} gained by a single
electron accelerating from rest though an electric potential difference of 1 volt in vacuum.

\begin{tabularx}{\linewidth}{|Y|} \hline
    $1 \; \si{\electronvolt} = 1.602 \times 10^{-19}$ \si{\joule} \\ \hline
\end{tabularx}

\begin{tabularx}{\linewidth}{|l|X|l|} \hline
    \multicolumn{3}{|c|}{$h \nu = \Phi + \text{KE}_{\text{max}}$} \\ \hline
    & \textbf{value} & \textbf{unit} \\ \hline
    $h$ & Planck's constant \par $= 6.626 \times 10^{-34}$& J/s \\
    $\nu$ & frequency & $s^{-1}$ or Hz \\ \hdashline
    $\Phi$ & work function of the metal & eV \\
    $\text{KE}_{\text{max}}$ & maximum kinetic energy of the ejected electron & eV \\ \hline
\end{tabularx}

Note that $\text{KE}_{\text{max}}$ \textit{cannot be negative}, so its minimum value is 0.

\subsection{Solar Spectrum}
The \textbf{electromagnetic spectrum} comprises of, in increasing energy:
\begin{enumerate}
    \item radio waves,
    \item microwaves,
    \item infrared,
    \item visible,
    \item ultraviolet,
    \item X-rays,
    \item gamma ($\gamma$) rays,
\end{enumerate}

White light disperses into a continuous spectrum in the visible spectrum.

However, sunlight is \emph{not continuous} --- there are \emph{dark lines} in
the \textbf{solar spectrum} due to the composition of the sun and the Earth's atmosphere.

\subsubsection{Aside: Spectroscopy}

\emph{The study of the interaction of \emph{electromagnetic radiation}
with \emph{matter} to obtain information about the identity and structure of substances.}

\textbf{Absorption spectroscopy} reveals that a colored object reflects light of that color
and absorbs light of the \textit{opposite color}.

\subsubsection{Emission Spectrums}
Energy can excite electrons of atoms from their \textbf{ground state} to an \textbf{excited state}.

In the excited state, electrons have higher energy, and when electrons are relaxed to their
ground state, their energy is released as light.

Therefore, the dark lines in the solar spectrum occur due to the absorption of energy
by the atoms in the sun's atmosphere.

\subsubsection{Rydberg Formula}
\emph{Used to calculate the wavelengths of a spectral line in hydrogenic species, i.e. H, $\text{He}^{\text{+}}$, $\text{Li}^{\text{2+}}$, etc.}

It generalizes the Balmer series ($n = 2$) for the visible region, the Lyman series ($n = 1$)
for the UV region, and the Paschen series ($n = 3$) for the IR region.

\begin{tabularx}{\linewidth}{|l|X|l|} \hline
    \multicolumn{3}{|c|}{$\frac{1}{\lambda} = R_H \left( \frac{1}{{n_1}^2} - \frac{1}{{n_2}^2} \right)$} \\ \hline
    & \textbf{value} & \textbf{unit} \\ \hline
    $\lambda$ & wavelength of light emitted & m \\ \hdashline
    $R_H$ & Rydberg constant $= 1.096776 \times 10^7$ & $\si{\per\metre}$ \\
    $n_1$ & final electron state (lower energy) $< n_2$ & \code{int} \\
    $n_2$ & initial electron state (higher energy) & \code{int} \\ \hline
\end{tabularx}